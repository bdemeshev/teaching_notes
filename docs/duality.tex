% Options for packages loaded elsewhere
\PassOptionsToPackage{unicode}{hyperref}
\PassOptionsToPackage{hyphens}{url}
\PassOptionsToPackage{dvipsnames,svgnames,x11names}{xcolor}
%
\documentclass[
  letterpaper,
  DIV=11,
  numbers=noendperiod]{scrartcl}

\usepackage{amsmath,amssymb}
\usepackage{iftex}
\ifPDFTeX
  \usepackage[T1]{fontenc}
  \usepackage[utf8]{inputenc}
  \usepackage{textcomp} % provide euro and other symbols
\else % if luatex or xetex
  \usepackage{unicode-math}
  \defaultfontfeatures{Scale=MatchLowercase}
  \defaultfontfeatures[\rmfamily]{Ligatures=TeX,Scale=1}
\fi
\usepackage{lmodern}
\ifPDFTeX\else  
    % xetex/luatex font selection
\fi
% Use upquote if available, for straight quotes in verbatim environments
\IfFileExists{upquote.sty}{\usepackage{upquote}}{}
\IfFileExists{microtype.sty}{% use microtype if available
  \usepackage[]{microtype}
  \UseMicrotypeSet[protrusion]{basicmath} % disable protrusion for tt fonts
}{}
\makeatletter
\@ifundefined{KOMAClassName}{% if non-KOMA class
  \IfFileExists{parskip.sty}{%
    \usepackage{parskip}
  }{% else
    \setlength{\parindent}{0pt}
    \setlength{\parskip}{6pt plus 2pt minus 1pt}}
}{% if KOMA class
  \KOMAoptions{parskip=half}}
\makeatother
\usepackage{xcolor}
\setlength{\emergencystretch}{3em} % prevent overfull lines
\setcounter{secnumdepth}{5}
% Make \paragraph and \subparagraph free-standing
\ifx\paragraph\undefined\else
  \let\oldparagraph\paragraph
  \renewcommand{\paragraph}[1]{\oldparagraph{#1}\mbox{}}
\fi
\ifx\subparagraph\undefined\else
  \let\oldsubparagraph\subparagraph
  \renewcommand{\subparagraph}[1]{\oldsubparagraph{#1}\mbox{}}
\fi


\providecommand{\tightlist}{%
  \setlength{\itemsep}{0pt}\setlength{\parskip}{0pt}}\usepackage{longtable,booktabs,array}
\usepackage{calc} % for calculating minipage widths
% Correct order of tables after \paragraph or \subparagraph
\usepackage{etoolbox}
\makeatletter
\patchcmd\longtable{\par}{\if@noskipsec\mbox{}\fi\par}{}{}
\makeatother
% Allow footnotes in longtable head/foot
\IfFileExists{footnotehyper.sty}{\usepackage{footnotehyper}}{\usepackage{footnote}}
\makesavenoteenv{longtable}
\usepackage{graphicx}
\makeatletter
\def\maxwidth{\ifdim\Gin@nat@width>\linewidth\linewidth\else\Gin@nat@width\fi}
\def\maxheight{\ifdim\Gin@nat@height>\textheight\textheight\else\Gin@nat@height\fi}
\makeatother
% Scale images if necessary, so that they will not overflow the page
% margins by default, and it is still possible to overwrite the defaults
% using explicit options in \includegraphics[width, height, ...]{}
\setkeys{Gin}{width=\maxwidth,height=\maxheight,keepaspectratio}
% Set default figure placement to htbp
\makeatletter
\def\fps@figure{htbp}
\makeatother

\usepackage{mathtools}
\newcommand{\cN}{\mathcal{N}}

\let\P\relax
\DeclareMathOperator{\P}{\mathbb{P}}
\DeclareMathOperator{\E}{\mathbb{E}}
\DeclareMathOperator{\plim}{plim}
\DeclareMathOperator{\Var}{Var}
\DeclareMathOperator{\Cov}{Cov}
\DeclareMathOperator{\Corr}{Corr}

\DeclarePairedDelimiter{\norm}{\lVert}{\rVert}
\DeclarePairedDelimiter{\abs}{\lvert}{\rvert}
\DeclarePairedDelimiter{\scalp}{\langle}{\rangle}
% TODO: automatic scaling does not work in pdf???

% TODO: no cyrillic font in section titles!
\setmainfont{Linux Libertine}

%\newfontfamily{\cyrillicfont}{Linux Libertine}
%\newfontfamily{\cyrillicfonttt}{Linux Libertine}
%\newfontfamily{\cyrillicfontsf}{Linux Libertine}
%\newfontfamily{\cyrillicfontrm}{Linux Libertine}
%\newfontfamily{\cyrillicfontbf}{Linux Libertine}
%\newfontfamily{\cyrillicfontmd}{Linux Libertine}
%\newfontfamily{\cyrillicfontit}{Linux Libertine}
%\newfontfamily{\cyrillicfontsl}{Linux Libertine}
%\newfontfamily{\cyrillicfontsc}{Linux Libertine}
%\newfontfamily{\cyrillicfontup}{Linux Libertine}
\KOMAoption{captions}{tableheading}
\makeatletter
\@ifpackageloaded{tcolorbox}{}{\usepackage[skins,breakable]{tcolorbox}}
\@ifpackageloaded{fontawesome5}{}{\usepackage{fontawesome5}}
\definecolor{quarto-callout-color}{HTML}{909090}
\definecolor{quarto-callout-note-color}{HTML}{0758E5}
\definecolor{quarto-callout-important-color}{HTML}{CC1914}
\definecolor{quarto-callout-warning-color}{HTML}{EB9113}
\definecolor{quarto-callout-tip-color}{HTML}{00A047}
\definecolor{quarto-callout-caution-color}{HTML}{FC5300}
\definecolor{quarto-callout-color-frame}{HTML}{acacac}
\definecolor{quarto-callout-note-color-frame}{HTML}{4582ec}
\definecolor{quarto-callout-important-color-frame}{HTML}{d9534f}
\definecolor{quarto-callout-warning-color-frame}{HTML}{f0ad4e}
\definecolor{quarto-callout-tip-color-frame}{HTML}{02b875}
\definecolor{quarto-callout-caution-color-frame}{HTML}{fd7e14}
\makeatother
\makeatletter
\@ifpackageloaded{caption}{}{\usepackage{caption}}
\AtBeginDocument{%
\ifdefined\contentsname
  \renewcommand*\contentsname{Содержание}
\else
  \newcommand\contentsname{Содержание}
\fi
\ifdefined\listfigurename
  \renewcommand*\listfigurename{Список Иллюстраций}
\else
  \newcommand\listfigurename{Список Иллюстраций}
\fi
\ifdefined\listtablename
  \renewcommand*\listtablename{Список Таблиц}
\else
  \newcommand\listtablename{Список Таблиц}
\fi
\ifdefined\figurename
  \renewcommand*\figurename{Рисунок}
\else
  \newcommand\figurename{Рисунок}
\fi
\ifdefined\tablename
  \renewcommand*\tablename{Таблица}
\else
  \newcommand\tablename{Таблица}
\fi
}
\@ifpackageloaded{float}{}{\usepackage{float}}
\floatstyle{ruled}
\@ifundefined{c@chapter}{\newfloat{codelisting}{h}{lop}}{\newfloat{codelisting}{h}{lop}[chapter]}
\floatname{codelisting}{Список}
\newcommand*\listoflistings{\listof{codelisting}{Список Каталогов}}
\makeatother
\makeatletter
\makeatother
\makeatletter
\@ifpackageloaded{caption}{}{\usepackage{caption}}
\@ifpackageloaded{subcaption}{}{\usepackage{subcaption}}
\makeatother
\ifLuaTeX
\usepackage[bidi=basic]{babel}
\else
\usepackage[bidi=default]{babel}
\fi
\babelprovide[main,import]{russian}
% get rid of language-specific shorthands (see #6817):
\let\LanguageShortHands\languageshorthands
\def\languageshorthands#1{}
\ifLuaTeX
  \usepackage{selnolig}  % disable illegal ligatures
\fi
\usepackage{bookmark}

\IfFileExists{xurl.sty}{\usepackage{xurl}}{} % add URL line breaks if available
\urlstyle{same} % disable monospaced font for URLs
\hypersetup{
  pdftitle={Двойственность},
  pdflang={ru},
  colorlinks=true,
  linkcolor={blue},
  filecolor={Maroon},
  citecolor={Blue},
  urlcolor={Blue},
  pdfcreator={LaTeX via pandoc}}

\title{Двойственность}
\author{}
\date{}

\begin{document}
\maketitle

\renewcommand*\contentsname{Содержание}
{
\hypersetup{linkcolor=}
\setcounter{tocdepth}{2}
\tableofcontents
}
\section{Двойственная
функция}\label{ux434ux432ux43eux439ux441ux442ux432ux435ux43dux43dux430ux44f-ux444ux443ux43dux43aux446ux438ux44f}

\begin{tcolorbox}[enhanced jigsaw, breakable, opacityback=0, title=\textcolor{quarto-callout-note-color}{\faInfo}\hspace{0.5em}{Двойственная функция}, opacitybacktitle=0.6, toptitle=1mm, bottomtitle=1mm, colframe=quarto-callout-note-color-frame, leftrule=.75mm, colback=white, titlerule=0mm, rightrule=.15mm, arc=.35mm, coltitle=black, colbacktitle=quarto-callout-note-color!10!white, toprule=.15mm, bottomrule=.15mm, left=2mm]

Двойственная функция в точке \(x^*\) показывает, насколько минимум надо
опустить прямую \(\langle x, x^* \rangle\) вниз, чтобы она целиком
оказалась ниже функции \(f(x)\).

\[
f^*(x^*) = \inf_t \{t \mid \langle x, x^* \rangle - t \leq f(x), \text{ для любого } x\}
\]

\end{tcolorbox}

Заметим, что определение легко переформулировать как поиск максимальной
разницы от функции \(f(x)\) вверх до прямой \(\langle x, x^* \rangle\).
Если в какой-то точке \(x\) прямая \(\langle x, x^* \rangle\) лежит
очень высоко относительно функции \(f(x)\), то и минимальная величина,
на которую придется опустить прямую очень велика.

\begin{tcolorbox}[enhanced jigsaw, breakable, opacityback=0, title=\textcolor{quarto-callout-note-color}{\faInfo}\hspace{0.5em}{Двойственная функция}, opacitybacktitle=0.6, toptitle=1mm, bottomtitle=1mm, colframe=quarto-callout-note-color-frame, leftrule=.75mm, colback=white, titlerule=0mm, rightrule=.15mm, arc=.35mm, coltitle=black, colbacktitle=quarto-callout-note-color!10!white, toprule=.15mm, bottomrule=.15mm, left=2mm]

Двойственная функция в точке \(x^*\) показывает, насколько минимум надо
опустить прямую \(\langle x, x^* \rangle\) вниз, чтобы она целиком
оказалась ниже функции \(f(x)\).

\[
f^*(x^*) = \sup_x \{\langle x, x^* \rangle - f(x)\}
\]

\end{tcolorbox}

Конечно, двойственная функция может принимать особое значение
\(\infty\).

Примеры.

\begin{tcolorbox}[enhanced jigsaw, breakable, opacityback=0, title=\textcolor{quarto-callout-note-color}{\faInfo}\hspace{0.5em}{Теоремка}, opacitybacktitle=0.6, toptitle=1mm, bottomtitle=1mm, colframe=quarto-callout-note-color-frame, leftrule=.75mm, colback=white, titlerule=0mm, rightrule=.15mm, arc=.35mm, coltitle=black, colbacktitle=quarto-callout-note-color!10!white, toprule=.15mm, bottomrule=.15mm, left=2mm]

\[
f^{**}(x) \leq f(x)
\]

\end{tcolorbox}

\begin{tcolorbox}[enhanced jigsaw, breakable, opacityback=0, title=\textcolor{quarto-callout-warning-color}{\faExclamationTriangle}\hspace{0.5em}{Доказательство неравенства}, opacitybacktitle=0.6, toptitle=1mm, bottomtitle=1mm, colframe=quarto-callout-warning-color-frame, leftrule=.75mm, colback=white, titlerule=0mm, rightrule=.15mm, arc=.35mm, coltitle=black, colbacktitle=quarto-callout-warning-color!10!white, toprule=.15mm, bottomrule=.15mm, left=2mm]

Для доказательства теоремы предположим, что \(f(x) \leq a\), и докажем,
что в этом случае \(f^{**}(x) \leq a\).

На старт: \[
f(x) \leq a
\]

Превратим левую и правую часть во что-то похожее на определение
двойственной функции, рассмотрев произвольные угловые коэффициенты
\(x^*\). \[
\langle x, x^* \rangle - f(x) \geq \langle x, x^* \rangle - a
\]

Левая часть не упадёт, если мы в ней вместо конкретного \(x\) возьмём
супремум: \[
\sup_{w}\langle w, x^* \rangle - f(w) \geq \langle x, x^* \rangle - a
\]

Это в чистом виде определение \(f^*(x^*)\): \[
f^*(x^*) \geq \langle x, x^* \rangle - a
\] Вспомним, что вектор наклонов \(x^*\) был произвольным!

Получается, что для любого вектора \(x^*\): \[
a \geq \langle x, x^* \rangle - f^*(x^*)
\]

Если \(a\) выше каждого возможного значения правой части, то \(a\) выше
максимально возможного. \[
a \geq \sup_{x^*}\{\langle x, x^* \rangle - f^*(x^*)\}
\]

А это и есть искомое утверждение: \[
a \geq f^{**}(x).
\]

\end{tcolorbox}

Для хороших выпуклых функций, оказывается, что \(f(x)=f^{**}(x)\).

\begin{tcolorbox}[enhanced jigsaw, breakable, opacityback=0, title=\textcolor{quarto-callout-note-color}{\faInfo}\hspace{0.5em}{Теоремка}, opacitybacktitle=0.6, toptitle=1mm, bottomtitle=1mm, colframe=quarto-callout-note-color-frame, leftrule=.75mm, colback=white, titlerule=0mm, rightrule=.15mm, arc=.35mm, coltitle=black, colbacktitle=quarto-callout-note-color!10!white, toprule=.15mm, bottomrule=.15mm, left=2mm]

Если \ldots., то \[
f^{**}(x) = f(x)
\]

\end{tcolorbox}

\begin{tcolorbox}[enhanced jigsaw, breakable, opacityback=0, title=\textcolor{quarto-callout-warning-color}{\faExclamationTriangle}\hspace{0.5em}{Доказательство случая равенства}, opacitybacktitle=0.6, toptitle=1mm, bottomtitle=1mm, colframe=quarto-callout-warning-color-frame, leftrule=.75mm, colback=white, titlerule=0mm, rightrule=.15mm, arc=.35mm, coltitle=black, colbacktitle=quarto-callout-warning-color!10!white, toprule=.15mm, bottomrule=.15mm, left=2mm]

В доказательстве теоремки о неравенстве есть только один неравносильный
переход. Обратим на него внимание.

\end{tcolorbox}

\begin{tcolorbox}[enhanced jigsaw, breakable, opacityback=0, title=\textcolor{quarto-callout-note-color}{\faInfo}\hspace{0.5em}{Правильное определение двойственной задачи}, opacitybacktitle=0.6, toptitle=1mm, bottomtitle=1mm, colframe=quarto-callout-note-color-frame, leftrule=.75mm, colback=white, titlerule=0mm, rightrule=.15mm, arc=.35mm, coltitle=black, colbacktitle=quarto-callout-note-color!10!white, toprule=.15mm, bottomrule=.15mm, left=2mm]

Рассмотрим задачу поиска минимума функции \(\phi\) по \(x\) с параметром
\(y\). \[
\min_x \phi(x, y)
\] Параметр \(y\) мы воспринимаем как возмущение, отклонение задачи от
некоторой исходной.

Исходная задача без возмущения имеет вид: \[
\min_x \phi(x, 0)
\]

Двойственной задачей назовём задачу: \[
\min_{y^*} \phi^*(0, y^*)
\]

\end{tcolorbox}

Определим функцию \(h(y)\) как оптимум исходной задачи про возмущении
\(y\), \[
h(y) = \min_x \phi(x, y)
\]

В этом случае исходную задачу без возмущения можно представить как
вычисление \(h(0)\).

\ldots{}

Таким образом, двойственная задача --- это вычисление \(h^{**}(0)\).

Помимо ещё одной абстрактной формулировки:

\begin{tcolorbox}[enhanced jigsaw, breakable, opacityback=0, title=\textcolor{quarto-callout-note-color}{\faInfo}\hspace{0.5em}{Правильное определение двойственной задачи}, opacitybacktitle=0.6, toptitle=1mm, bottomtitle=1mm, colframe=quarto-callout-note-color-frame, leftrule=.75mm, colback=white, titlerule=0mm, rightrule=.15mm, arc=.35mm, coltitle=black, colbacktitle=quarto-callout-note-color!10!white, toprule=.15mm, bottomrule=.15mm, left=2mm]

Рассмотрим задачу поиска минимума функции \(\phi\) по \(x\) с параметром
\(y\). \[
h(y) = \min_x \phi(x, y)
\]

Прямая задача --- это поиск \(h(0)\).

Двойственная задача --- это поиск \(h^{**}(0)\).

\end{tcolorbox}

Вспомним, что \(h^{**}(x) \leq h(x)\) для всех функций и
\(h^{**}(x) = h(x)\) для хороших \ldots{} функций.

Следовательно, мы бесплатно получаем две теоремки:

\begin{tcolorbox}[enhanced jigsaw, breakable, opacityback=0, title=\textcolor{quarto-callout-note-color}{\faInfo}\hspace{0.5em}{Теоремка}, opacitybacktitle=0.6, toptitle=1mm, bottomtitle=1mm, colframe=quarto-callout-note-color-frame, leftrule=.75mm, colback=white, titlerule=0mm, rightrule=.15mm, arc=.35mm, coltitle=black, colbacktitle=quarto-callout-note-color!10!white, toprule=.15mm, bottomrule=.15mm, left=2mm]

Если \(h(y) = \min_x \phi(x, y)\) --- оптимум возмущённой задачи, то \[
h^{**}(0) = \min_{y^*} \phi^*(0, y^*) \leq \min_x \phi(x, 0) = h(0).
\]

\end{tcolorbox}

\begin{tcolorbox}[enhanced jigsaw, breakable, opacityback=0, title=\textcolor{quarto-callout-note-color}{\faInfo}\hspace{0.5em}{Теоремка}, opacitybacktitle=0.6, toptitle=1mm, bottomtitle=1mm, colframe=quarto-callout-note-color-frame, leftrule=.75mm, colback=white, titlerule=0mm, rightrule=.15mm, arc=.35mm, coltitle=black, colbacktitle=quarto-callout-note-color!10!white, toprule=.15mm, bottomrule=.15mm, left=2mm]

Если \(h(y) = \min_x \phi(x, y)\) --- оптимум возмущённой задачи, и
\ldots, то \[
h^{**}(0) = \min_{y^*} \phi^*(0, y^*) = \min_x \phi(x, 0) = h(0).
\]

\end{tcolorbox}

Источники:



\end{document}
